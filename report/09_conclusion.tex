\section{Conclusion}
\label{sec:conclusion}

In this work, we built a Markovian model for the Aloha \ac{MAC} protocol.
We started from a quick overview of the protocol and the theoretical available model.
We build a very simple model based on a \ac{CTMC}, then we refined it to better represent the real behaviour of the system.
Finally, we compared the models with the simulator results.

The proposed model is able to predict the behaviour of Aloha in terms of throughput and packets collision rate for an arbitrary number of station and different transmission rates.
Our analysis shows that networks with a high number of stations achieve a slightly higher throughput at low load, but a considerably smaller one at high loads than networks with few nodes.
This is confirmed by the collision rate, low for small loads and very closed to $1$ for high loads.

Modelling a system is fundamental to get a good understanding of its behaviour and predict its evolution.
Models are cheap but give very useful indications.
Models should be used, together with simulators, in the design of each complex in order to achieve the desired goals with a high confidence.
