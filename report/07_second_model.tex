\section{Second Model}
\label{sec:second_model}

In the previous model, state $1$ can be reached from either state $0$ or state $2$.
In the second case, the packet is very likely to be corrupted, since it was on the channel with another packet.
We shall refine the model to distinguish those cases.
We replace state $1$ with two new states: $1G$ ($1$ good) and $1B$ ($1$ bad).
State $1G$ can be reached only from state $0$, whereas state $1B$ can be reached only from state $2$.
\cref{fig:model2} shows the resulting \ac{MC} for $n=3$.

\begin{figure}[t!]
    \centering
    \begin{tikzpicture}[->, >=stealth', auto, semithick, node distance=2cm]
        \tikzstyle{every state}=[fill=white,draw=black,thick,text=black,scale=1]
        
        \node[state]    (0)                     {$0$};
        \node[state]    (G)[above right of=0]   {$1G$};
        \node[state]    (B)[below right of=0]   {$1B$};
        \node[state]    (2)[below right of=G]   {$2$};
        \node[state]    (3)[right of=2]         {$3$};
        
        \path
            (0) edge[bend left]    node{$3\lambda$}     (G)
            (G) edge[bend left]    node{$2\lambda$}     (2)
            (B) edge[bend left]    node{$2\lambda$}     (2)
            (2) edge[bend left]    node{$\lambda$}      (3)
            (3) edge[bend left]    node{$3\mu$}         (2)
            (2) edge[bend left]    node{$2\mu$}         (B)
            (B) edge[bend left]    node{$\mu$}          (0)
            (G) edge[bend left]    node{$\mu$}          (0)
        ;
    \end{tikzpicture}
    \caption{Markov Chain that represent the second model for $3$ stations. The state represent the number of nodes on the channel.}
    \label{fig:model2}
\end{figure}

As for the previous model, the \ac{CTMC} admits a steady state solution, which can be computed as follows:
\begin{equation*}
    \begin{cases}
        \pi_i = \frac{\binom{n}{i} \lambda^i \mu^{n-i}}{(\lambda + \mu)^n}, & \mbox{if } i \neq 1 \\[1em]
        \pi_{1G} = \frac{n \lambda \mu^n}{(\lambda + \mu)^n((n-1)\lambda + \mu)}, \\[1em]
        \pi_{1B} = \frac{n(n-1) \lambda^2 \mu^{n-1}}{(\lambda + \mu)^n((n-1)\lambda + \mu)}. \\
    \end{cases}
\end{equation*}

Note that $\pi_{1G} + \pi_{1B}$ equals to $\pi_1$ of the previous model.

Similarly to the previous model, throughout and collision rate are computed as follows:
\begin{equation*}
    throughput = \pi_{1G} \cdot r \cdot \frac{n-1}{n}\,,
\end{equation*}

\begin{equation*}
    collision\_rate = \frac{\pi_{1B} + \sum_{i = 2}^{n} \pi_i}{\pi_{1G} + \pi_{1B} + \sum_{i = 2}^{n} \pi_i}\,.
\end{equation*}

As for the first model, we assume that stations transmit packets as soon as they get them, so they do not have any queue.
