\section{Introduction}
\label{sec:introduction}

Models are powerful and widely used tools to analyze, compare and evaluate the performances of a system.
Together with simulations, they allow to take the right decisions in the design and construction of a system and guarantee that it behaves as expected.
Models predict the behaviour of a system as a function of some parameters, allowing a very cheap and effective ``what-if'' analysis.

In this assignment, we will build and present a Markovian Model for the Aloha \ac{MAC} protocol.
We will briefly discuss the behaviour of Aloha. Then, we will present the assumptions we started from and show how we build and refined the model step-by-step.
Finally, we will compare the model's predictions with the the simulator's results.
