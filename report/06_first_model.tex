\section{First Model}
\label{sec:first_model}

The first model uses a \ac{CTMC} to represent the state of the channel.
Each node indicate the number of packets currently on the channel.
At the beginning there are zero packets.
At any moment, all station have an equal probability to start to transmit at any given moment: the transition from state $0$ to state $1$ has a rate of $n \lambda$.
If there is $1$ packet on the channel, one station is already transmitting and it will not be able to transmit another packet at the same time.
Thus, the rate to move from state $1$ to state $2$ is given by $(n-1) \lambda$.
For the same reason, the transition rate from state $k$ to $k+1$ is given by $(n - k) \lambda$.

Each packet needs some time to be transmitted.
Since we assumed an exponential distribution for the sizes, we define $\mu$ as follows:
\begin{equation*}
    \mu = \frac{1}{t_{TX}}
    = \Big( \frac{s}{r} \Big) ^ {-1}
    = \frac{r}{s},
\end{equation*}
where $t_{TX}$ denotes the transmission time for one averaged sized packet.
Similarly to the $\lambda$ case, we define the transition rate from state $k$ to $k-1$ as $k\mu$.
For example, if we are in state $2$ (there are $2$ packets on the channel), we have a transition rate to state $1$ of $2\mu$.

\begin{figure}[t!]
    \centering
    \begin{tikzpicture}[->, >=stealth', auto, semithick, node distance=2cm]
        \tikzstyle{every state}=[fill=white,draw=black,thick,text=black,scale=1]
        
        \node[state]    (0)                 {$0$};
        \node[state]    (1)[right of=0]     {$1$};
        \node[state]    (2)[right of=1]     {$2$};
        \node[state]    (3)[right of=2]     {$3$};
        
        \path
            (0) edge[bend left]    node{$3\lambda$}     (1)
            (1) edge[bend left]    node{$2\lambda$}     (2)
            (2) edge[bend left]    node{$\lambda$}      (3)
            (3) edge[bend left]    node{$3\mu$}         (2)
            (2) edge[bend left]    node{$2\mu$}         (1)
            (1) edge[bend left]    node{$\mu$}          (0)
        ;
    \end{tikzpicture}
    \caption{Markov Chain that represent the first model for $3$ stations. The state represent the number of nodes on the channel.}
    \label{fig:simple}
\end{figure}

Since the \ac{CTMC} is time-homogeneous and has only recurrent-positive states, it admits a steady state solution $\pi$ that can be computed as:
\begin{equation*}
    \begin{cases}
        \sum_{i=0}^{n} \pi_i = 1 \\[0.5em]
        \pi Q = 0
    \end{cases}
\end{equation*}

$Q = [q_{ij}]$ is the infinitesimal generator matrix that represent the transitions rate from state $i$ to $j$ of the chain.
For any \ac{CTMC}, the diagonal $q_{ii}$ of the matrix $Q$ is defined as:

\begin{equation*}
    q_{ii} = - \!\!\! \sum_{j = 0, j \neq i}^{\infty} q_{ij}.
\end{equation*}

For $n=3$ showed in \cref{fig:simple} we have:
\begin{equation*}
Q =
\begin{bmatrix}
    -3\lambda   & 3\lambda            & 0                   & 0         \\
    \mu         & (-2\lambda - \mu)   & 2\lambda            & 0         \\
    0           & 2\mu                & (-\lambda - 2\mu)   & \lambda   \\
    0           & 0                   & 3\mu                & -3\mu
\end{bmatrix}
\end{equation*}
\begin{equation*}
    \begin{cases}
        \pi_0 + \pi_1 + \pi_2 + \pi_3 = 1 \\
        -3\lambda\pi_0 + \mu\pi_1 = 0 \\
        3\lambda\pi_0 + (-\lambda-2\mu) + 2\mu\pi_2 = 0 \\
        2\lambda\pi_1 + (-\lambda-2\mu)\pi_2 + 3\mu\pi_3 = 0 \\
        \lambda\pi_2 - 3\mu\pi_3 = 0
    \end{cases}
\end{equation*}

which holds the solution:
\begin{equation*}
    \begin{aligned}[l]
        \pi_0 = \frac{\mu^3}{(\lambda + \mu)^3}, \\[1em]
        \pi_1 = \frac{3\lambda\mu^2}{(\lambda + \mu)^3}, \\
    \end{aligned}
    \;\;\;\;\;
    \begin{aligned}[l]
        \pi_2 = \frac{3\lambda^2\mu}{(\lambda + \mu)^3}, \\[1em]
        \pi_3 = \frac{\lambda^3}{(\lambda + \mu)^3}.
    \end{aligned}
\end{equation*}

It is easy to show that the solution of this model for an arbitrary number of stations $n$ is given by:
\begin{equation*}
    \pi_i = \frac{\binom{n}{i} \lambda^i \mu^{n-i}}{(\lambda + \mu)^n}.
\end{equation*}

We are interested in the probability $\pi_1$ that there is only $1$ packet on the channel, since it is the only case when a packet can be transmitted without collisions.
Given $\pi_1$, we can compute the average throughput at the receiver as:
\begin{equation*}
    throughput = \pi_1 \cdot r \cdot \frac{n-1}{n}.
\end{equation*}

The term $\frac{n-1}{n}$ is given by the definition of throughput as the size of the received packets over the total size of emitted packets.
If we consider a single station, it receives packets from the other $n-1$ ones: the packets sent by itself are not counted in its throughput.

In a similar way, we can compute the collision rate as the probability of having $2$ or more packets in the air over the probability of at least $1$ packet:
\begin{equation*}
    collision\_rate = \frac{\sum_{i = 2}^{n} \pi_i}{\sum_{i = 1}^{n} \pi_i}.
\end{equation*}

The model does not consider the stations' packets queue: it assumes the stations to transmit each packet as soon as they receive it from the application.
