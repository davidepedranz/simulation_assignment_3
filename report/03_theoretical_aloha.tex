\section{Aloha Analysis}
\label{sec:theoretical_aloha}

A theoretical analysis of the Aloha protocol has been proposed by Tanenbaum in 2003 \cite{tanenbaum2003computernetworks}.
The model is derived from the analysis of packets collisions assuming that packets follow a Poisson process.
The predicted throughput is computed as:
\begin{equation*}
    throughput = G \cdot e^{-2G} \cdot r\,,
\end{equation*}
where $G$ is the normalized load, i.e. the average number of packets transmitted for each unit of time, and $r$ denotes transmission rate of each station \cite{wiki:aloha}.
$G$ can be computed as the total offered load over the transmission rate:
\begin{equation*}
    G = \frac{load}{r}\,.
\end{equation*}
